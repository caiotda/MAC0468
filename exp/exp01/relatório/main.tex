% MODELO PARA SER USADO NOS RELATÓRIOS

\documentclass[10pt,a4paper,twocolumn]{article} % 10pt font size (11 and 12 also possible), A4 paper (letterpaper for US letter) and two column layout (remove for one column)

\usepackage[brazil]{babel} % escrevendo em português
\usepackage[T1]{fontenc} % Output font encoding for international characters
\usepackage[utf8]{inputenc} % Required for inputting international characters
\usepackage{hyperref}
\usepackage{amsmath,amsfonts,amsthm} % Math packages for equations
\usepackage{graphicx} % Required for adding images

\usepackage{geometry} % Required for adjusting page dimensions

\geometry{
	top=1cm, % Top margin
	bottom=1.5cm, % Bottom margin
	left=2cm, % Left margin
	right=2cm, % Right margin
	%includehead, % Include space for a header
	%includefoot, % Include space for a footer
	%showframe, % Uncomment to show how the type block is set on the page
}

\setlength{\columnsep}{7mm} % Column separation width

%----------------------------------------------------------------------------------------
%	Relatório: cabeçalho
%----------------------------------------------------------------------------------------

\title{Modelo para escrever o seu relatório} 

\author{  SEU NOME E CO-AUTORES\\
    Laboratório de Visão Computacional - IME - USP\\
}
\date{}
%----------------------------------------------------------------------------------------

\begin{document}

\maketitle % Print the title

%\thispagestyle{empty} % Apply the page style for the first page (no headers and footers)

%----------------------------------------------------------------------------------------
%	ABSTRACT
%----------------------------------------------------------------------------------------

\begin{abstract}
    Escreva aqui um resumo geral do relatório descrevendo o problema,
    os objetivos, a solução, a metodologia de avaliação, e os principais resultados.
    Esse resumo deve conter não mais que 1000 caracteres.
    O relatório inteiro deve ter 2 páginas, incluindo figuras, tabelas, etc.
    Você pode incluir uma 3a página contendo apenas referências bibliográficas.
\end{abstract}

%----------------------------------------------------------------------------------------
%	CONTEÚDO
%----------------------------------------------------------------------------------------

\section{Introdução}

Você pode usar o \href{https://www.overleaf.com}{Overleaf} para criar e compilar seus relatórios online. Caso você nunca tenha usado \LaTeX{} dê uma olhada \href{https://www.overleaf.com/learn}{nesse link}.

Esse é um exemplo de como incluir uma citação em seu texto \cite{Ref1}. Não deixe de ler sobre o \href{https://www.ctan.org/pkg/}{biblatex}~\cite{Biblatex}. Veja o arquivo example.bib para incluir suas referências.

Esse é um exemplo de expressão matemática:

\begin{equation}
	A = 
	\begin{bmatrix}
		A_{11} & A_{21} \\
		A_{21} & A_{22}
	\end{bmatrix}
\end{equation}

%------------------------------------------------
% Use % para seus comentários
%

\subsection{Subseção}

Você pode usar subseções, mas lembre-se de limitar seu relatório a 2 páginas.

O itemize é útil para listar elementos.

\begin{itemize}
	\item item 1 da lista
	\item segundo 
	\item mais um
\end{itemize}

Assim como o enumerate:
\begin{enumerate}
	\item item 1 do enumerate
	\item outro
	\item mais um
\end{enumerate}

Tabelas como a tabela~\ref{tab:Tabela1} podem ser muito importantes em um relatório:

\begin{table}[hbt]
	\centering
    \begin{tabular}{ll|r}\hline
        coluna 1 & coluna 2 & total \\\hline
        x & y & z\\
        1 & 2 & 3\\
    \end{tabular}
	\caption{Uma tabela}
    \label{tab:Tabela1}
\end{table}

%------------------------------------------------

\section{Use figuras!}

Assim se cria uma referência (no caso para a figura) \ref{fig:gato}.

\begin{figure}
	\includegraphics[width=0.9\linewidth]{cat011.png} % Figure image
	\caption{Gatinho} % Figure caption
	\label{fig:gato} % Label for referencing with \ref{bear}
\end{figure}


%----------------------------------------------------------------------------------------
%	BIBLIOGRAPHY
%----------------------------------------------------------------------------------------
\bibliographystyle{plain}
\begin{thebibliography}{9}
    \bibitem{Ref1} \emph{First and last \LaTeX{} example.},
    John Doe 50 B.C. 
    \bibitem{Biblatex} \emph{Não deixe de ver sobre o biblatex.} \href{https://www.ctan.org/pkg/}{biblatex}. 
\end{thebibliography}
    
%----------------------------------------------------------------------------------------

\end{document}
